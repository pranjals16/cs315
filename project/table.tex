\documentclass{article}
\usepackage[margin=0.5in]{geometry}
\usepackage{parskip}
\usepackage{epsfig}
\usepackage[latin1]{inputenc}
\usepackage[normalem]{ulem}

%\special{landscape}
\evensidemargin-1.5cm
\oddsidemargin-1.5cm

\pagestyle{plain}

\title{\huge{CS315:PROJECT\\Hall Management System\\Phase 3(Relation Design)}}
\author{Pranjal Singh - 10511\\   \\ \{spranjal\}@iitk.ac.in}
\date{\today}
\begin{document}
\maketitle
\section{Tables}


\subsection{Entity Sets}

\begin{itemize}
\item Students (\uline{Email},Name,AccNo,Dept,HallNo,Rollno,Roomno,IP,PhoneNo,Laptop,BloodGrp,HomeAddr,GuardianName,\\MessBill,CanteenBill,Courier,ShopBill)
\item StudentIssues (\uline{cid},CompEmail,StudComplaint)
\item Employees (\uline{EmpID},Designation,PhoneNo,EmpEmail,AccNo,Name,Address)
\item EmpIssues (\uline{ccid},CompEmpID,EmpComplaint)
\end{itemize}

\subsection{Relationships}

\begin{itemize}
\item havethese (\uline{Studcid},StudEmail)
\item have (\uline{Empcid},EEmpID)
\end{itemize}


\section{Normalization}
The above set of tables are already in normalized form.\\
All the underlined items in each table functionally determine all other items and each underlined item is a \emph{superkey}. Hence relation is in BCNF.\\
Initially in ER Diagram, there were 2 tables namely Students and StudDetails which have been merged into one due to 1-1 relationship.

\section{Integrity Constraints}

\subsection{Students}
Email:Primary Key

\subsection{StudentIssues}
cid:Primary Key\\
CompEmail: Foreign Key(Email from Students Table)

\subsection{Employees}
EmpID:Primary Key

\subsection{EmpIssues}
ccid:Primary Key\\
CompEmpID: Foreign Key(EmpID from Employees Table)

\subsection{havethese}
Studcid:Primary Key, Foreign Key(cid from StudentIssues Table)\\
StudEmail: Foreign Key(Email from Students Table)

\subsection{have}
Empcid:Primary Key, Foreign Key(ccid from EmpIssues Table)\\
EEmpID: Foreign Key(EmpID from Employees Table)

\end{document}
